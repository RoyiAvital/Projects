% !TeX document-id = {d3b263d9-4ae8-4edf-a203-72d775f66a9e}
\documentclass[]{article}
\usepackage{amsmath}
\usepackage{amsfonts}
\usepackage{amssymb}
\usepackage{amsthm}
\usepackage{dsfont} % /mathds{}
\usepackage{graphicx}
\usepackage{float}
\usepackage{hyperref} % Hyperlinks
\usepackage{xcolor} % \colorbox{} & \textcolor{}

%% \usepackage{minted} % Syntax Highlighting (Requires Python)
%% !TeX TXS-program:compile = txs:///pdflatex/[--shell-escape]

% See https://tex.stackexchange.com/a/78393/1661
\usepackage[framed, numbered, autolinebreaks, useliterate]{mcode} % Syntax Highlighting MATLAB

% Math Symbols
% http://www.cs.put.poznan.pl/ksiek/latexmath.html

%\newtheorem{theorem}{Theorem}[section]
%\newtheorem{corollary}{Corollary}[theorem]
%\newtheorem*{lemma*}[theorem]{Lemma}
\newtheorem*{lemma}{Lemma}
\newtheorem*{remark}{Remark}

\DeclareMathOperator{\sign}{sign}
\DeclareMathOperator{\prox}{prox}
\DeclareMathOperator{\diag}{diag}
\DeclareMathOperator{\Tr}{Tr}

% Math Commands
\newcommand{\MyParen}[1]{\left( #1 \right)}
\newcommand{\MyBrack}[1]{\left\lbrack #1 \right\rbrack}
\newcommand{\MyBrace}[1]{\left\lbrace #1 \right\rbrace}
\newcommand{\MyNorm}[2]{{\left\| #1 \right\|}_{#2}}
\newcommand{\MyNormSqr}[2]{{\left\| #1 \right\|}_{#2}^{2}}
\newcommand{\MyNormTwo}[1]{\MyNorm{#1}{2}}
\newcommand{\MyNormTwoSqr}[1]{\MyNormSqr{#1}{2}}
\newcommand{\MyAbs}[1]{\left| #1 \right|}
\newcommand{\MyCeil}[1]{\left \lceil #1 \right \rceil}
\newcommand{\MyFloor}[1]{\left \lfloor #1 \right \rfloor}
\newcommand{\MyProd}[2]{\langle #1, #2 \rangle}
\newcommand{\MyUndBrace}[2]{\underset{#2}{\underbrace{#1}}}
% \newcommand{\RR}[1]{\mathds{R}^{#1}} % Asaf's Style
\newcommand{\RR}[1]{\mathbb{R}^{#1}}
\newcommand{\EE}[1]{\mathbb{E} \MyBrack{#1}}

% Text Commands
\newcommand{\inlinecode}[1]{\colorbox{lightgray}{\texttt{#1}}}

%opening
\title{Projection onto Balls}
\author{Royi Avital}

\begin{document}
	
	\maketitle
	
%	\begin{abstract}
%		Notes on Math, Engineering, etc...
%	\end{abstract}

	\tableofcontents
	
	\newpage
	
	\section{Projection onto the $ {L}_{1} $ Ball}
	% https://math.stackexchange.com/questions/2327504
	\begin{align*}
	\arg \min_{ \MyNorm{x}{1} \leq r } \MyBrace{ \frac{1}{2} \MyNormTwoSqr{ x - y } }
	\end{align*}
	
	The Lagrangian of the problem can be written as:
	
	
	\begin{align*}
	L \left( x, \lambda \right) & = \frac{1}{2} {\left\| x - y \right\|}^{2} + \lambda \left( {\left\| x \right\|}_{1} - r \right) && \text{} \\
	& = \sum_{i = 1}^{n} \left( \frac{1}{2} { \left( {x}_{i} - {y}_{i} \right) }^{2} + \lambda \left| {x}_{i} \right| \right) - \lambda r && \text{Component wise form}
	\end{align*}
	
	The Dual Function is given by:
	
	\begin{align*}
	g \left( \lambda \right) = \inf_{x} L \left( x, \lambda \right)
	\end{align*}
	
	The above can be solved component wise for the term $ \left( \frac{1}{2} { \left( {x}_{i} - {y}_{i} \right) }^{2} + \lambda \left| {x}_{i} \right| \right) $ which is solved by the \href{https://en.wikipedia.org/wiki/Proximal_gradient_methods_for_learning#Solving_for_.7F.27.22.60UNIQ--postMath-0000002A-QINU.60.22.27.7F_proximity_operator}{Soft Thresholding Operator}:
	
	\begin{align*}
	{x}_{i}^{\ast} = \sign \left( {y}_{i} \right) { \left( \left| {y}_{i} \right| - \lambda \right) }_{+}
	\end{align*}
	
	Where $ {\left( t \right)}_{+} = \max \left( t, 0 \right) $.
	
	Clearly, when $ \lambda = 0 $ it suggests that $ x = y $ and $ \MyNorm{y}{1} \leq r $. For the case $ \lambda > 0 $, which suggests $ \MyNorm{y}{1} > 1 $, all needed is to find the optimal $ \lambda > 0 $ which is given by the root of the objective function (Which is the constrain of the KKT System):
	
	\begin{align*}
	h \left( \lambda \right) & = \sum_{i = 1}^{n} \left| {x}_{i}^{\ast} \left( \lambda \right) \right| - r \\
	& = \sum_{i = 1}^{n} { \left( \left| {y}_{i} \right| - \lambda \right) }_{+} - r
	\end{align*}
	
	Namely, the solution is given by $ {\lambda}^{\ast} = \lambda : h \MyParen{\lambda} = 0 $. The above is a Piece Wise linear function of $ \lambda $ and its Derivative given by:
	
	\begin{align*}
	\frac{\mathrm{d} }{\mathrm{d} \lambda} h \left( \lambda \right) & = \frac{\mathrm{d} }{\mathrm{d} \lambda} \sum_{i = 1}^{n} { \left( \left| {y}_{i} \right| - \lambda \right) }_{+} \\
	& = \sum_{i = 1}^{n} -{ \mathbf{1} }_{\left\{ \left| {y}_{i} \right| - \lambda > 0 \right\}}
	\end{align*}
	
	Hence it can be solved using Newton Iteration.
	
	\begin{lstlisting}[caption={MATLAB Code - $ {L}_{1} $ Ball Projection}, label={lst:codeL1Projection}]
	function [ vX ] = ProjectL1Ball( vY, ballRadius, stopThr )
	% ----------------------------------------------------------------------------------------------- %
	% [ vX ] = ProjectL1Ball( vY, ballRadius, stopThr )
	%   Solving the Orthoginal Porjection Problem of the input vector onto the
	%   L1 Ball using Dual Function and Newtin Iteration.
	% Input:
	%   - vY            -   Input Vector.
	%                       Structure: Vector (Column).
	%                       Type: 'Single' / 'Double'.
	%                       Range: (-inf, inf).
	%   - ballRadius    -   Ball Radius.
	%                       Sets the Radiuf of the L1 Ball. For Unit L1 Ball
	%                       set to 1.
	%                       Structure: Scalar.
	%                       Type: 'Single' / 'Double'.
	%                       Range: (0, inf).
	%   - stopThr       -   Stopping Threshold.
	%                       Sets the trheold of the Newton Iteration. The
	%                       absolute value of the Objective Function will be
	%                       below the threshold.
	%                       Structure: Scalar.
	%                       Type: 'Single' / 'Double'.
	%                       Range: (0, inf).
	% Output:
	%   - vX            -   Output Vector.
	%                       The projection of the Input Vector onto the Simplex
	%                       Ball.
	%                       Structure: Vector (Column).
	%                       Type: 'Single' / 'Double'.
	%                       Range: (-inf, inf).
	% References
	%   1.  https://math.stackexchange.com/questions/2327504.
	%   2.  https://en.wikipedia.org/wiki/Newton%27s_method.
	% Remarks:
	%   1.  a
	% TODO:
	%   1.  U.
	% Release Notes:
	%   -   1.0.001     29/06/2017  Royi Avital
	%       *   Enforcing Lambda to be non negative (Dealing with the case 'vY'
	%           is obeying || vY ||_1 <= ballRadius).
	%   -   1.0.000     27/06/2017  Royi Avital
	%       *   First release version.
	% ----------------------------------------------------------------------------------------------- %
	
	FALSE   = 0;
	TRUE    = 1;
	
	OFF     = 0;
	ON      = 1;
	
	paramLambda     = 0;
	% The objective functions which its root (The 'paramLambda' which makes it
	% vanish) is the solution
	objVal          = sum(max(abs(vY) - paramLambda, 0)) - ballRadius;
	
	while(abs(objVal) > stopThr)
	objVal          = sum(max(abs(vY) - paramLambda, 0)) - ballRadius;
	df              = sum(-((abs(vY) - paramLambda) > 0)); %<! Derivative of 'objVal' with respect to Lambda
	paramLambda     = paramLambda - (objVal / df); %<! Newton Iteration
	end
	
	% Enforcing paramLambda >= 0. Otherwise it suggests || vY ||_1 <= ballRadius.
	% Hence the Optimal vX is given by vX = vY.
	paramLambda = max(paramLambda, 0);
	
	vX = sign(vY) .* max(abs(vY) - paramLambda, 0);
	
	
	end
	
	\end{lstlisting}
	
	\section{Projection onto the $ {L}_{2} $ Ball}
	
	\begin{align*}
		\arg \min_{ \MyNormTwo{x} \leq r } \MyBrace{ \frac{1}{2} \MyNormTwoSqr{ x - y } }
	\end{align*}
	
	The Lagrangian is given by:
	
	\begin{align*}
	L \MyParen{x, \lambda} = \frac{1}{2} \MyNormTwoSqr{ x - y } + \lambda \MyParen{ {x}^{T} x - r }
	\end{align*}
	
	The problem above is convex and Slater's condition holds by choosing $ x = \boldsymbol{0} $.
	
	The KKT Conditions are given by:
	
	\begin{align*}
	{\nabla}_{x} L \MyParen{ x, \lambda } = x - y + 2 \lambda x & = 0 && \text{(1) Stationary} \\
	\lambda \MyParen{ {x}^{T} x  - r} 	& = 0 		&& \text{(2) Slackness} \\
	{x}^{T} x  - r						& \leq 0 	&& \text{(3) Primal Feasibility} \\
	\lambda 							& \geq 0 	&& \text{(4) Dual Feasibility} \\
	\end{align*}
	
	From (1) one could see that if $ \lambda = 0 $ then $ x = y $ and from (3) it means $ {x}^{T} x = {y}^{T} y \leq r $. If $ \lambda > 0 $ then from (2) it means $ {x}^{T} x = r $.
	Taking (1) and multiply it by $ {x}^{T} $ yields (Remember $ {x}^{T} x = r $):
	
	$$ {x}^{T} x - {x}^{T} y + 2 \lambda {x}^{T} x = 0 \Rightarrow \lambda = \frac{ {x}^{T} y - r }{2 r} $$
	
	Plugging the result back into (1) yields $ x = \MyParen{1 + \frac{{x}^{T} y - r}{r}}^{-1} y = \frac{r}{ {x}^{T} y } y $. Since $ {x}^{T} x = r $ and $ x $ is a scaled version of $ y $ (Namely has the same direction as $ y $) which results in:
	
	$$ x = \begin{cases}
	y 							& \text{ if } \MyNormTwo{y} \leq r \\ 
	r \frac{y}{\MyNormTwo{y}} 	& \text{ if } \MyNormTwo{y} > r 
	\end{cases} $$
	
	\begin{lstlisting}[caption={MATLAB Code - $ {L}_{2} $ Ball Projection}, label={lst:codeL2Projection}]
	function [ vX ] = ProjectL2Ball( vY, ballRadius )
	% ----------------------------------------------------------------------------------------------- %
	% [ vX ] = ProjectL2Ball( vY, ballRadius, stopThr )
	%   Solving the Orthoginal Porjection Problem of the input vector onto the
	%   L1 Ball.
	% Input:
	%   - vY            -   Input Vector.
	%                       Structure: Vector (Column).
	%                       Type: 'Single' / 'Double'.
	%                       Range: (-inf, inf).
	%   - ballRadius    -   Ball Radius.
	%                       Sets the Radius of the L2 Ball. For Unit L2 Ball
	%                       set to 1.
	%                       Structure: Scalar.
	%                       Type: 'Single' / 'Double'.
	%                       Range: (0, inf).
	% Output:
	%   - vX            -   Output Vector.
	%                       The projection of the Input Vector onto the L2
	%                       Ball.
	%                       Structure: Vector (Column).
	%                       Type: 'Single' / 'Double'.
	%                       Range: (-inf, inf).
	% References
	%   1.  h
	% Remarks:
	%   1.  a
	% TODO:
	%   1.  U.
	% Release Notes:
	%   -   1.0.000     29/06/2017  Royi Avital
	%       *   First release version.
	% ----------------------------------------------------------------------------------------------- %
	
	FALSE   = 0;
	TRUE    = 1;
	
	OFF     = 0;
	ON      = 1;
	
	vX = min((ballRadius / norm(vY, 2)), 1) * vY;
	
	
	end
	
	\end{lstlisting}
	
	\section{Projection onto the $ {L}_{\infty} $ Ball}
	
	\begin{align*}
	\arg \min_{ \MyNorm{x}{\infty} \leq r } \MyBrace{ \frac{1}{2} \MyNormTwoSqr{ x - y } }
	\end{align*}
	
	Since $ \MyNorm{x}{\infty} = \max_{i} \MyAbs{ {x}_{i} }, \; i = 1, 2, \ldots, n $ the above can be written as:
	
	\begin{alignat*}{3}
	\arg \min_{x} 		& \quad && \frac{1}{2} \MyNormTwoSqr{ x - y } 	& \quad 	& \text{} \\
	\text{subject to} 	& \quad && {x}_{i} \leq r						& i = 1, 2, \ldots, n 	& \text{} \\
						& \quad -&& {x}_{i} \leq r						&  i = 1, 2, \ldots, n	& \text{}
	\end{alignat*}
	
	The Lagrangian is given by:
	
	\begin{align*}
	L \MyParen{x, {\lambda}_{1}, {\lambda}_{2}} = \frac{1}{2} \MyNormTwoSqr{ x - y } + {\lambda}_{1}^{T} \MyParen{ x - r } + {\lambda}_{2}^{T} \MyParen{ -x - r }
	\end{align*}
	
	The problem above is convex and Slater's condition holds by choosing $ x = \boldsymbol{0} $.
	
	The KKT Conditions are given by:
	
	\begin{align*}
	{\nabla}_{x} L \MyParen{x, {\lambda}_{1}, {\lambda}_{2}} = x - y + {\lambda}_{1} x - {\lambda}_{2} x & = 0 && \text{(1) Stationary} \\
	{\lambda}_{1, i} \MyParen{ {x}_{i}  - r} 	& = 0 		&& \text{(2) Slackness} \\
	{\lambda}_{2, i} \MyParen{ -{x}_{i}  - r} 	& = 0 		&& \text{(3) Slackness} \\
	{x}_{i}  - r						& \leq 0 	&& \text{(4) Primal Feasibility} \\
	-{x}_{i}  - r						& \leq 0 	&& \text{(5) Primal Feasibility} \\
	{\lambda}_{1, i}, {\lambda}_{2, i} 							& \geq 0 	&& \text{(6) Dual Feasibility} \\
	\end{align*}
	
	As can be seen from above, the problem can be solved component wise. Hence the sub script $ i $ for the Lagrange Multiplier will be be neglected. From (2) and (3) it can be shown that either $ {\lambda}_{1} > 0 $ or $ {\lambda}_{2} > 0 $ but not both. Hence if $ {\lambda}_{1} = 0 $ then $ {\lambda}_{2} = 0 $ which means $ {x}_{i} = {y}_{i} $ and $ \MyAbs{ {x}_{i} } \leq 1 $.  Moreover, if $  {\lambda}_{1} > 0 $ then $ {\lambda}_{2} = 0 $ hence $ {x}_{i} = r $ and $ {y}_{i} > 1 $. The same goes the other way around which yields:
	
	$$ {x}_{i} = \sign \MyParen{ {y}_{i} } \min \MyBrace{ r, \MyAbs{ {y}_{i} } }, \; i = 1, 2, \ldots, n $$
	
	\begin{lstlisting}[caption={MATLAB Code - $ {L}_{\infty} $ Ball Projection}, label={lst:codeLInfProjection}]
	function [ vX ] = ProjectLInfBall( vY, ballRadius )
	% ----------------------------------------------------------------------------------------------- %
	% [ vX ] = ProjectL2Ball( vY, ballRadius, stopThr )
	%   Solving the Orthoginal Porjection Problem of the input vector onto the
	%   L Inf Ball.
	% Input:
	%   - vY            -   Input Vector.
	%                       Structure: Vector (Column).
	%                       Type: 'Single' / 'Double'.
	%                       Range: (-inf, inf).
	%   - ballRadius    -   Ball Radius.
	%                       Sets the Radius of the L Inf Ball. For Unit L Inf 
	%                       Ball set to 1.
	%                       Structure: Scalar.
	%                       Type: 'Single' / 'Double'.
	%                       Range: (0, inf).
	% Output:
	%   - vX            -   Output Vector.
	%                       The projection of the Input Vector onto the L Inf
	%                       Ball.
	%                       Structure: Vector (Column).
	%                       Type: 'Single' / 'Double'.
	%                       Range: (-inf, inf).
	% References
	%   1.  h
	% Remarks:
	%   1.  a
	% TODO:
	%   1.  U.
	% Release Notes:
	%   -   1.0.000     29/06/2017  Royi Avital
	%       *   First release version.
	% ----------------------------------------------------------------------------------------------- %
	
	FALSE   = 0;
	TRUE    = 1;
	
	OFF     = 0;
	ON      = 1;
	
	vX = sign(vY) .* min(abs(vY), ballRadius);
	
	
	end
	
	\end{lstlisting}
	
	\section{Projection onto the Simplex}
	
	\begin{align*}
	\arg \min_{ x \succeq 0, \boldsymbol{1}^{T} x = r } \MyBrace{ \frac{1}{2} \MyNormTwoSqr{ x - y } }
	\end{align*}
	
	The Lagrangian of the problem can be written as (Leaving the non negativity constrain implicit):
	
	
	\begin{align*}
	L \left( x, \mu \right) & = \frac{1}{2} {\left\| x - y \right\|}^{2} + \mu \left( \boldsymbol{1}^{T} x - r \right) && \text{} \\
	\end{align*}
	
	The Dual Function is given by (Includes the Non Negativity Constrain):
	
	\begin{align*}
	g \left( \mu \right) & = \inf_{x \succeq 0} L \left( x, \mu \right) && \text{} \\
	& = \inf_{x \succeq 0} \sum_{i = 1}^{n} \left( \frac{1}{2} { \left( {x}_{i} - {y}_{i} \right) }^{2} + \mu {x}_{i} \right) - \mu r && \text{Component wise form} \\
	& = \inf_{x \succeq 0} \frac{1}{2} \sum_{i = 1}^{n} \left( {x}_{i} - \left( {y}_{i} - \mu \right) \right)^{2} + \mu \left( \boldsymbol{1}^{T} y - r \right) + n {\mu}^{2}
	\end{align*}
	
	The minimization with respect to $ {x}_{i} $ is basically a projection problem into $ \mathbb{R}_{+} $ of $ {y}_{i} - \mu $ which is given by:
	
	\begin{align*}
	{x}_{i}^{\ast} = { \left( {y}_{i} - \mu \right) }_{+}
	\end{align*}
	
	The solution is given by finding the $ \mu $ which holds the equality constrain. In the $ {L}_{1} $ case above the constrain was in inequality form hence $ \lambda $ had to be non negative Yet the above is equality constrain hence $ \mu $ can have any value and it is not limited to non negativity as $ \lambda $ in the $ {L}_{1} $ case.
	
	The function, From the KKT, which enforces equality, is given by:
	
	\begin{align*}
	h \left( \mu \right) = \sum_{i = 1}^{n} {x}_{i}^{\ast} - r & = \sum_{i = 1}^{n} { \left( {y}_{i} - \mu \right) }_{+} - r
	\end{align*}
	
	Namely, the solution is given by $ {\mu}^{\ast} = \mu : h \MyParen{\mu} = 0 $. The above is a Piece Wise linear function of $ \mu $ and its Derivative given by:
	
	\begin{align*}
	\frac{\mathrm{d} }{\mathrm{d} \mu} h \left( \mu \right) & = \frac{\mathrm{d} }{\mathrm{d} \mu} \sum_{i = 1}^{n} { \left( {y}_{i} - \mu \right) }_{+} \\
	& = \sum_{i = 1}^{n} -{ \mathbf{1} }_{\left\{ {y}_{i} - \mu > 0 \right\}}
	\end{align*}
	
	Hence it can be solved using Newton Iteration.
	
	\begin{lstlisting}[caption={MATLAB Code - Simplex Projection}, label={lst:codeSimplexProjection}]
	function [ vX ] = ProjectSimplex( vY, ballRadius, stopThr )
	% ----------------------------------------------------------------------------------------------- %
	% [ vX ] = ProjectSimplex( vY, ballRadius, stopThr )
	%   Solving the Orthoginal Porjection Problem of the input vector onto the
	%   Simplex Ball using Dual Function and Newton Iteration.
	% Input:
	%   - vY            -   Input Vector.
	%                       Structure: Vector (Column).
	%                       Type: 'Single' / 'Double'.
	%                       Range: (-inf, inf).
	%   - ballRadius    -   Ball Radius.
	%                       Sets the Radius of the Simplex Ball. For Unit
	%                       Simplex set to 1.
	%                       Structure: Scalar.
	%                       Type: 'Single' / 'Double'.
	%                       Range: (0, inf).
	%   - stopThr       -   Stopping Threshold.
	%                       Sets the trheolds of the Newton Iteration. The
	%                       absolute value of the Objective Function will be
	%                       below the threshold.
	%                       Structure: Scalar.
	%                       Type: 'Single' / 'Double'.
	%                       Range: (0, inf).
	% Output:
	%   - vX            -   Output Vector.
	%                       The projection of the Input Vector onto the Simplex
	%                       Ball.
	%                       Structure: Vector (Column).
	%                       Type: 'Single' / 'Double'.
	%                       Range: (-inf, inf).
	% References
	%   1.  https://math.stackexchange.com/questions/2327504.
	%   2.  https://en.wikipedia.org/wiki/Newton%27s_method.
	% Remarks:
	%   1.  a
	% TODO:
	%   1.  U.
	% Release Notes:
	%   -   1.0.001     09/05/2017  Royi Avital
	%       *   Renaming 'paramLambda' -> 'paramMu' to match derivation.
	%   -   1.0.000     09/05/2017  Royi Avital
	%       *   First release version.
	% ----------------------------------------------------------------------------------------------- %
	
	FALSE   = 0;
	TRUE    = 1;
	
	OFF     = 0;
	ON      = 1;
	
	% Choosing paramMu = min(vY) - ballRadius yields starting value of paramMu
	% which ensures the objective value to have positive value -> Easier to
	% find its root.
	paramMu = min(vY) - ballRadius;
	% The objective functions which its root (The 'paramMu' which makes it
	% vanish) is the solution
	objFun      = sum( max(vY - paramMu, 0) ) - ballRadius;
	
	while(abs(objFun) > stopThr)
	    objFun      = sum( max(vY - paramMu, 0) ) - ballRadius;
	    df          = sum(-((vY - paramMu) > 0)); %<! Derivative of 'objVal' with respect to Mu
	    paramMu     = paramMu - (objFun / df); %<! Newton Iteration
	end
	
	vX = max(vY - paramMu, 0);
	
	
	end
	
	\end{lstlisting}
	
	
\end{document}
